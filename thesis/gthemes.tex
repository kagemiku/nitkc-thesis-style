%%%%%%%%%%%%
% 卒業論文用
%
% 使用方法:
% プリアンブルの最後で
% %%%%%%%%%%%%
% 卒業論文用
%
% 使用方法:
% プリアンブルの最後で
% %%%%%%%%%%%%
% 卒業論文用
%
% 使用方法:
% プリアンブルの最後で
% %%%%%%%%%%%%
% 卒業論文用
%
% 使用方法:
% プリアンブルの最後で
% \input{gthemes}
%
%
%%%%%%%%%%%%
% 上:(1inch+0pt)+0pt+20pt+15pt=37.7mm
% 下:293mm-37.6mm-225mm=30.4mm
%
\setlength{\textheight}{225truemm}
%
% 1行40文字
% 左:(1inch+0pt)+10mm=35.4
% 右:210mm-40x11pt
%
\setlength{\textwidth}{40zw}
%\setlength{\hoffset}{10truemm}
\setlength{\oddsidemargin}{10truemm}
\setlength{\evensidemargin}{10truemm}
%
% 42行になるように行送りを調整
%
\renewcommand{\baselinestretch}{0.86}

\和暦

%%%
% 表紙作成のコマンド
%  \hyoshi{タイトル}{学籍番号}{氏名}{指導教員名}
%%%
\makeatletter
\newcommand{\hyoshi}[4]{\newpage%
\begin{center}
\vspace*{6em}
{\Large 平成27年度 卒業研究\par}\vspace*{1em}
{\LARGE\textbf{#1}\par}\vskip 30em {\large \lineskip 1em
独立行政法人 国立高等専門学校機構\par
鹿児島工業高等専門学校\par
情報工学科\par
\vskip 1em
\makebox[5zw][r]{研究者} : \makebox[10zw][l]{#3(#2)}\par
\makebox[5zw][r]{指導教員} : \makebox[10zw][l]{#4}\par
\makebox[5zw][r]{提出日} : \makebox[10zw][l]{\@date}\par}
\end{center}
%\par
%\vskip 1.5em
\thispagestyle{empty}
\newpage}
\makeatother

%%%
% 目次作成のコマンド
%  \mokuji
%%%
\newcommand{\mokuji}{\newpage%
\pagenumbering{roman}
\tableofcontents
\clearpage
\pagenumbering{arabic}}



\makeindex

%
%
%%%%%%%%%%%%
% 上:(1inch+0pt)+0pt+20pt+15pt=37.7mm
% 下:293mm-37.6mm-225mm=30.4mm
%
\setlength{\textheight}{225truemm}
%
% 1行40文字
% 左:(1inch+0pt)+10mm=35.4
% 右:210mm-40x11pt
%
\setlength{\textwidth}{40zw}
%\setlength{\hoffset}{10truemm}
\setlength{\oddsidemargin}{10truemm}
\setlength{\evensidemargin}{10truemm}
%
% 42行になるように行送りを調整
%
\renewcommand{\baselinestretch}{0.86}

\和暦

%%%
% 表紙作成のコマンド
%  \hyoshi{タイトル}{学籍番号}{氏名}{指導教員名}
%%%
\makeatletter
\newcommand{\hyoshi}[4]{\newpage%
\begin{center}
\vspace*{6em}
{\Large 平成27年度 卒業研究\par}\vspace*{1em}
{\LARGE\textbf{#1}\par}\vskip 30em {\large \lineskip 1em
独立行政法人 国立高等専門学校機構\par
鹿児島工業高等専門学校\par
情報工学科\par
\vskip 1em
\makebox[5zw][r]{研究者} : \makebox[10zw][l]{#3(#2)}\par
\makebox[5zw][r]{指導教員} : \makebox[10zw][l]{#4}\par
\makebox[5zw][r]{提出日} : \makebox[10zw][l]{\@date}\par}
\end{center}
%\par
%\vskip 1.5em
\thispagestyle{empty}
\newpage}
\makeatother

%%%
% 目次作成のコマンド
%  \mokuji
%%%
\newcommand{\mokuji}{\newpage%
\pagenumbering{roman}
\tableofcontents
\clearpage
\pagenumbering{arabic}}



\makeindex

%
%
%%%%%%%%%%%%
% 上:(1inch+0pt)+0pt+20pt+15pt=37.7mm
% 下:293mm-37.6mm-225mm=30.4mm
%
\setlength{\textheight}{225truemm}
%
% 1行40文字
% 左:(1inch+0pt)+10mm=35.4
% 右:210mm-40x11pt
%
\setlength{\textwidth}{40zw}
%\setlength{\hoffset}{10truemm}
\setlength{\oddsidemargin}{10truemm}
\setlength{\evensidemargin}{10truemm}
%
% 42行になるように行送りを調整
%
\renewcommand{\baselinestretch}{0.86}

\和暦

%%%
% 表紙作成のコマンド
%  \hyoshi{タイトル}{学籍番号}{氏名}{指導教員名}
%%%
\makeatletter
\newcommand{\hyoshi}[4]{\newpage%
\begin{center}
\vspace*{6em}
{\Large 平成27年度 卒業研究\par}\vspace*{1em}
{\LARGE\textbf{#1}\par}\vskip 30em {\large \lineskip 1em
独立行政法人 国立高等専門学校機構\par
鹿児島工業高等専門学校\par
情報工学科\par
\vskip 1em
\makebox[5zw][r]{研究者} : \makebox[10zw][l]{#3(#2)}\par
\makebox[5zw][r]{指導教員} : \makebox[10zw][l]{#4}\par
\makebox[5zw][r]{提出日} : \makebox[10zw][l]{\@date}\par}
\end{center}
%\par
%\vskip 1.5em
\thispagestyle{empty}
\newpage}
\makeatother

%%%
% 目次作成のコマンド
%  \mokuji
%%%
\newcommand{\mokuji}{\newpage%
\pagenumbering{roman}
\tableofcontents
\clearpage
\pagenumbering{arabic}}



\makeindex

%
%
%%%%%%%%%%%%
% 上:(1inch+0pt)+0pt+20pt+15pt=37.7mm
% 下:293mm-37.6mm-225mm=30.4mm
%
\setlength{\textheight}{225truemm}
%
% 1行40文字
% 左:(1inch+0pt)+10mm=35.4
% 右:210mm-40x11pt
%
\setlength{\textwidth}{40zw}
%\setlength{\hoffset}{10truemm}
\setlength{\oddsidemargin}{10truemm}
\setlength{\evensidemargin}{10truemm}
%
% 42行になるように行送りを調整
%
\renewcommand{\baselinestretch}{0.86}

\和暦

%%%
% 表紙作成のコマンド
%  \hyoshi{タイトル}{学籍番号}{氏名}{指導教員名}
%%%
\makeatletter
\newcommand{\hyoshi}[4]{\newpage%
\begin{center}
\vspace*{6em}
{\Large 平成27年度 卒業研究\par}\vspace*{1em}
{\LARGE\textbf{#1}\par}\vskip 30em {\large \lineskip 1em
独立行政法人 国立高等専門学校機構\par
鹿児島工業高等専門学校\par
情報工学科\par
\vskip 1em
\makebox[5zw][r]{研究者} : \makebox[10zw][l]{#3(#2)}\par
\makebox[5zw][r]{指導教員} : \makebox[10zw][l]{#4}\par
\makebox[5zw][r]{提出日} : \makebox[10zw][l]{\@date}\par}
\end{center}
%\par
%\vskip 1.5em
\thispagestyle{empty}
\newpage}
\makeatother

%%%
% 目次作成のコマンド
%  \mokuji
%%%
\newcommand{\mokuji}{\newpage%
\pagenumbering{roman}
\tableofcontents
\clearpage
\pagenumbering{arabic}}



\makeindex
